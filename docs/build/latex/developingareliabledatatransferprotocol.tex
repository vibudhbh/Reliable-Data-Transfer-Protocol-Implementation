%% Generated by Sphinx.
\def\sphinxdocclass{report}
\documentclass[letterpaper,10pt,english]{sphinxmanual}
\ifdefined\pdfpxdimen
   \let\sphinxpxdimen\pdfpxdimen\else\newdimen\sphinxpxdimen
\fi \sphinxpxdimen=.75bp\relax
\ifdefined\pdfimageresolution
    \pdfimageresolution= \numexpr \dimexpr1in\relax/\sphinxpxdimen\relax
\fi
%% let collapsible pdf bookmarks panel have high depth per default
\PassOptionsToPackage{bookmarksdepth=5}{hyperref}

\PassOptionsToPackage{booktabs}{sphinx}
\PassOptionsToPackage{colorrows}{sphinx}

\PassOptionsToPackage{warn}{textcomp}
\usepackage[utf8]{inputenc}
\ifdefined\DeclareUnicodeCharacter
% support both utf8 and utf8x syntaxes
  \ifdefined\DeclareUnicodeCharacterAsOptional
    \def\sphinxDUC#1{\DeclareUnicodeCharacter{"#1}}
  \else
    \let\sphinxDUC\DeclareUnicodeCharacter
  \fi
  \sphinxDUC{00A0}{\nobreakspace}
  \sphinxDUC{2500}{\sphinxunichar{2500}}
  \sphinxDUC{2502}{\sphinxunichar{2502}}
  \sphinxDUC{2514}{\sphinxunichar{2514}}
  \sphinxDUC{251C}{\sphinxunichar{251C}}
  \sphinxDUC{2572}{\textbackslash}
\fi
\usepackage{cmap}
\usepackage[T1]{fontenc}
\usepackage{amsmath,amssymb,amstext}
\usepackage{babel}



\usepackage{tgtermes}
\usepackage{tgheros}
\renewcommand{\ttdefault}{txtt}



\usepackage[Bjarne]{fncychap}
\usepackage{sphinx}

\fvset{fontsize=auto}
\usepackage{geometry}


% Include hyperref last.
\usepackage{hyperref}
% Fix anchor placement for figures with captions.
\usepackage{hypcap}% it must be loaded after hyperref.
% Set up styles of URL: it should be placed after hyperref.
\urlstyle{same}

\addto\captionsenglish{\renewcommand{\contentsname}{Contents:}}

\usepackage{sphinxmessages}
\setcounter{tocdepth}{1}



\title{Developing a Reliable Data Transfer Protocol}
\date{Mar 20, 2025}
\release{}
\author{Vibudh Bhardwaj}
\newcommand{\sphinxlogo}{\vbox{}}
\renewcommand{\releasename}{}
\makeindex
\begin{document}

\ifdefined\shorthandoff
  \ifnum\catcode`\=\string=\active\shorthandoff{=}\fi
  \ifnum\catcode`\"=\active\shorthandoff{"}\fi
\fi

\pagestyle{empty}
\sphinxmaketitle
\pagestyle{plain}
\sphinxtableofcontents
\pagestyle{normal}
\phantomsection\label{\detokenize{index::doc}}


\sphinxstepscope


\chapter{Reliable Data Transfer (RDT) Protocol Documentation}
\label{\detokenize{documentation:reliable-data-transfer-rdt-protocol-documentation}}\label{\detokenize{documentation::doc}}

\section{Overview}
\label{\detokenize{documentation:overview}}
\sphinxAtStartPar
The Reliable Data Transfer (RDT) protocol implementation ensures reliable transmission of data over UDP, providing mechanisms such as sequence numbering, acknowledgment packets, checksums for integrity verification, retransmissions, and a sliding window approach for efficiency.


\section{Requirements}
\label{\detokenize{documentation:requirements}}\begin{itemize}
\item {} 
\sphinxAtStartPar
Python 3.7+

\item {} 
\sphinxAtStartPar
Standard libraries: socket, threading, struct, hashlib

\end{itemize}


\section{Usage}
\label{\detokenize{documentation:usage}}

\subsection{client.py}
\label{\detokenize{documentation:client-py}}
\begin{sphinxVerbatim}[commandchars=\\\{\}]
python\PYG{+w}{ }client.py\PYG{+w}{ }\PYGZlt{}file\PYGZus{}to\PYGZus{}send\PYGZgt{}\PYG{+w}{ }\PYGZlt{}client\PYGZus{}port\PYGZgt{}\PYG{+w}{ }\PYGZlt{}simulator\PYGZus{}port\PYGZgt{}
\end{sphinxVerbatim}


\subsection{server.py}
\label{\detokenize{documentation:server-py}}
\begin{sphinxVerbatim}[commandchars=\\\{\}]
python\PYG{+w}{ }server.py\PYG{+w}{ }\PYGZlt{}received\PYGZus{}filename\PYGZgt{}\PYG{+w}{ }\PYGZlt{}server\PYGZus{}port\PYGZgt{}
\end{sphinxVerbatim}


\subsection{simulator.py}
\label{\detokenize{documentation:simulator-py}}
\begin{sphinxVerbatim}[commandchars=\\\{\}]
python\PYG{+w}{ }simulator.py
\end{sphinxVerbatim}


\section{Command\sphinxhyphen{}Line Arguments}
\label{\detokenize{documentation:command-line-arguments}}

\subsection{client.py}
\label{\detokenize{documentation:id1}}

\begin{savenotes}\sphinxattablestart
\sphinxthistablewithglobalstyle
\centering
\begin{tabulary}{\linewidth}[t]{TT}
\sphinxtoprule
\sphinxstyletheadfamily 
\sphinxAtStartPar
Argument
&\sphinxstyletheadfamily 
\sphinxAtStartPar
Description
\\
\sphinxmidrule
\sphinxtableatstartofbodyhook
\sphinxAtStartPar
\sphinxcode{\sphinxupquote{\textless{}file\_to\_send\textgreater{}}}
&
\sphinxAtStartPar
File path to the data to be sent to the server.
\\
\sphinxhline
\sphinxAtStartPar
\sphinxcode{\sphinxupquote{\textless{}client\_port\textgreater{}}}
&
\sphinxAtStartPar
Local UDP port to bind the client socket.
\\
\sphinxhline
\sphinxAtStartPar
\sphinxcode{\sphinxupquote{\textless{}simulator\_port\textgreater{}}}
&
\sphinxAtStartPar
UDP port where the simulator is listening.
\\
\sphinxbottomrule
\end{tabulary}
\sphinxtableafterendhook\par
\sphinxattableend\end{savenotes}


\subsection{server.py}
\label{\detokenize{documentation:id2}}

\begin{savenotes}\sphinxattablestart
\sphinxthistablewithglobalstyle
\centering
\begin{tabulary}{\linewidth}[t]{TT}
\sphinxtoprule
\sphinxstyletheadfamily 
\sphinxAtStartPar
Argument
&\sphinxstyletheadfamily 
\sphinxAtStartPar
Description
\\
\sphinxmidrule
\sphinxtableatstartofbodyhook
\sphinxAtStartPar
\sphinxcode{\sphinxupquote{\textless{}received\_filename\textgreater{}}}
&
\sphinxAtStartPar
Name of the file to save received data.
\\
\sphinxhline
\sphinxAtStartPar
\sphinxcode{\sphinxupquote{\textless{}server\_port\textgreater{}}}
&
\sphinxAtStartPar
Local UDP port where the server listens for incoming packets.
\\
\sphinxbottomrule
\end{tabulary}
\sphinxtableafterendhook\par
\sphinxattableend\end{savenotes}


\section{Functionality}
\label{\detokenize{documentation:functionality}}

\subsection{Packet Class}
\label{\detokenize{documentation:packet-class}}\begin{itemize}
\item {} 
\sphinxAtStartPar
\sphinxstylestrong{Packet(seq\_num, data, ack)}: Represents a packet with a sequence number, data payload, acknowledgment flag, and checksum.
\sphinxhyphen{} \sphinxstylestrong{calculate\_checksum()}: Generates SHA\sphinxhyphen{}256 checksum.
\sphinxhyphen{} \sphinxstylestrong{to\_bytes()}: Serializes the packet for sending over UDP.
\sphinxhyphen{} \sphinxstylestrong{from\_bytes(bytes\_data)}: Deserializes bytes to reconstruct a packet.
\sphinxhyphen{} \sphinxstylestrong{is\_valid()}: Verifies packet integrity using checksum.

\end{itemize}


\subsection{RDT\_Sender Class}
\label{\detokenize{documentation:rdt-sender-class}}\begin{itemize}
\item {} 
\sphinxAtStartPar
\sphinxstylestrong{send(data)}: Sends data reliably, managing a sliding window and retransmissions.

\item {} 
\sphinxAtStartPar
\sphinxstylestrong{\_start\_timer()}: Starts a retransmission timer for packet timeouts.

\item {} 
\sphinxAtStartPar
\sphinxstylestrong{\_timeout\_handler()}: Handles retransmissions when packets timeout.

\item {} 
\sphinxAtStartPar
\sphinxstylestrong{\_recv\_ack\_thread(total\_chunks)}: Dedicated thread receiving ACK packets to advance the sliding window.

\item {} 
\sphinxAtStartPar
\sphinxstylestrong{close()}: Closes socket and cancels retransmission timer.

\end{itemize}


\subsection{RDT\_Receiver Class}
\label{\detokenize{documentation:rdt-receiver-class}}\begin{itemize}
\item {} 
\sphinxAtStartPar
\sphinxstylestrong{listen()}: Listens continuously for incoming packets, handles acknowledgments, and stores data.

\item {} 
\sphinxAtStartPar
\sphinxstylestrong{\_send\_ack(seq\_num, addr)}: Sends acknowledgment for received packets.

\item {} 
\sphinxAtStartPar
\sphinxstylestrong{get\_data()}: Assembles and returns the received data in correct order.

\item {} 
\sphinxAtStartPar
\sphinxstylestrong{close()}: Gracefully terminates the receiver.

\end{itemize}


\subsection{NetworkSimulator Class}
\label{\detokenize{documentation:networksimulator-class}}\begin{itemize}
\item {} 
\sphinxAtStartPar
Simulates packet loss, corruption, and delays to test the robustness of the RDT implementation.

\item {} 
\sphinxAtStartPar
Configurable parameters: \sphinxcode{\sphinxupquote{loss\_rate}}, \sphinxcode{\sphinxupquote{corruption\_rate}}, and \sphinxcode{\sphinxupquote{delay\_rate}}.

\end{itemize}


\section{Example Usage}
\label{\detokenize{documentation:example-usage}}

\subsection{1. Start the Simulator}
\label{\detokenize{documentation:start-the-simulator}}
\begin{sphinxVerbatim}[commandchars=\\\{\}]
python\PYG{+w}{ }simulator.py
\end{sphinxVerbatim}

\sphinxAtStartPar
Runs the simulator with default parameters (packet loss, delay, corruption).


\subsection{2. Start the Server}
\label{\detokenize{documentation:start-the-server}}
\begin{sphinxVerbatim}[commandchars=\\\{\}]
python\PYG{+w}{ }server.py\PYG{+w}{ }received\PYGZus{}file.txt\PYG{+w}{ }\PYG{l+m}{9001}
\end{sphinxVerbatim}

\sphinxAtStartPar
Starts the server to listen on port 9001 and save data as \sphinxcode{\sphinxupquote{received\_file.txt}}.


\subsection{3. Start the Client}
\label{\detokenize{documentation:start-the-client}}
\begin{sphinxVerbatim}[commandchars=\\\{\}]
python\PYG{+w}{ }client.py\PYG{+w}{ }file\PYGZus{}to\PYGZus{}send.txt\PYG{+w}{ }\PYG{l+m}{8000}\PYG{+w}{ }\PYG{l+m}{9000}
\end{sphinxVerbatim}

\sphinxAtStartPar
Sends \sphinxcode{\sphinxupquote{file\_to\_send.txt}} from client port 8000 through simulator at port 9000.


\section{Notes}
\label{\detokenize{documentation:notes}}\begin{itemize}
\item {} 
\sphinxAtStartPar
Ensure UDP ports 8000, 9000, and 9001 are open and not blocked by firewalls.

\item {} 
\sphinxAtStartPar
Adjust simulator parameters to reflect different network conditions.

\end{itemize}


\section{License}
\label{\detokenize{documentation:license}}
\sphinxAtStartPar
This implementation is intended for educational and demonstration purposes. Always ensure you have permission for testing network communications.

\sphinxAtStartPar
Author: Vibudh Bhardwaj
Last Updated: 2025\sphinxhyphen{}03\sphinxhyphen{}18



\renewcommand{\indexname}{Index}
\printindex
\end{document}